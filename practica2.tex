\documentclass[english]{article}
\usepackage[T1]{fontenc}
\usepackage[latin9]{inputenc}
\usepackage{graphicx}
\usepackage{babel}
\begin{document}
\title{Practica 2}
\author{Por Ruben Cazorla Rodriguez}
\maketitle

\section*{Lenguaje}

\noindent Dado un lenguaje L=\{a,b\} que solo contenga \textsl{``a''}

\section{{\LARGE{}Ejercicio 1}}

\subsection{{\Large{}Enunciado}}

Construye un DFA que reconozca el lenguaje y rechace las cadenas que
no cumplan con la condicion, y prueba el automata con 6 cadenas.

\subsection{{\Large{}Solucion}}

\subsubsection{{\large{}Construccion}}

Para crear el DFA hay que definir (K, $\sum$, $\delta$, s, F) donde:
\begin{itemize}
\item K=\{$q_{0},q_{1}q_{2}$\}, el conjunto de estados no vacios
\item $\sum$=\{a,b\}, el alfabeto
\item $\delta$=\{($q_{0},a,q_{1}$),($q_{0},b,q_{2}$),($q_{1},a,q_{2}$),($q_{1},b,q_{2}$),($q_{2},a,q_{2}$),($q_{2},b,q_{2}$)\},
funciones de transicion
\item s=$q_{0}$, el estado inicial
\item F=\{$q_{1}$\}, el conjunto de estados iniciales
\end{itemize}
Una vez definido el DFA, utilizando JFLAP, podemos esquematizarlo
graficamente:

\,
\begin{center}
\includegraphics{/home/alumno/Descargas/DFA}
\par\end{center}

\subsubsection{{\large{}Pruebas}}

Utilizando la opcion \textit{``multiple run'' }podemos probar las
6 cadenas para probar si el automata funciona correctamente:\textit{ }
\begin{itemize}
\item a : acepta
\item aaaa: rechaza
\item ab: rechaza
\item bbbaaa: rechaza
\item b: rechaza
\item baba: rechaza
\end{itemize}

\section{{\LARGE{}Ejercicio 2}}

\subsection{{\Large{}Enunciado}}

Abre en Octave el archivo \textit{``finiteautomata.m'' }, Especifica
en \textit{``finiteautomata.json''} el lenguaje del ejercicio 1
y prueba que funciona correctamente.

\subsection{{\Large{}Solucion}}

Primero abrimos \textit{``finiteautomata.json''} y definimos el
lenguaje en un json.

\textbf{\{ }

\textbf{\textquotedbl name\textquotedbl{} : \textquotedbl practica2\textquotedbl , }

\textbf{\textquotedbl representation\textquotedbl{} : \{ }

\textbf{\textquotedbl K\textquotedbl{} : {[}\textquotedbl q0\textquotedbl ,
\textquotedbl q1\textquotedbl , \textquotedbl q2\textquotedbl{]}, }

\textbf{\textquotedbl A\textquotedbl{} : {[}\textquotedbl a\textquotedbl ,\textquotedbl b\textquotedbl{]}, }

\textbf{\textquotedbl s\textquotedbl{} : \textquotedbl q0\textquotedbl , }

\textbf{\textquotedbl F\textquotedbl{} : {[}\textquotedbl q1\textquotedbl{]}, }

\textbf{\textquotedbl t\textquotedbl{} : {[}{[}\textquotedbl q0\textquotedbl ,
\textquotedbl a\textquotedbl , \textquotedbl q1\textquotedbl{]}, }

\textbf{{[}\textquotedbl q0\textquotedbl , \textquotedbl b\textquotedbl ,
\textquotedbl q2\textquotedbl{]}, }

\textbf{{[}\textquotedbl q1\textquotedbl , \textquotedbl a\textquotedbl ,
\textquotedbl q2\textquotedbl{]}, }

\textbf{{[}\textquotedbl q1\textquotedbl , \textquotedbl b\textquotedbl ,
\textquotedbl q2\textquotedbl{]}, }

\textbf{{[}\textquotedbl q2\textquotedbl , \textquotedbl a\textquotedbl ,
\textquotedbl q2\textquotedbl{]}, }

\textbf{{[}\textquotedbl q2\textquotedbl , \textquotedbl a\textquotedbl ,
\textquotedbl q2\textquotedbl{]}{]} \} }

\textbf{\}}

\noindent Y una vez definido, abrimos \textit{``finiteautomata.m''}
en Octave, y utilizando el comando\textit{ finiteautomata(``nombre\_automata'',''String'')}
mostrara el camino que sigue el automata analizando la cadena introducida
y mostrara si pertenece al lenguaje mediante una variable llamada
``ans'', la cual sera 1 si el lenguaje acepta y 0 si lo rechaza.

\noindent Sabiendo esto, y utilizando los ejemplos del ejercicio anterior,
sus resultados son los mismos que en el anterior ejercicio:
\begin{itemize}
\item a : acepta
\item aaaa: rechaza
\item ab: rechaza
\item bbbaaa: rechaza
\item b: rechaza
\item baba: rechaza
\end{itemize}

\end{document}
