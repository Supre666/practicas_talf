\documentclass[english]{article}
\usepackage[T1]{fontenc}
\usepackage[latin9]{inputenc}
\usepackage{setspace}
\usepackage{babel}
\begin{document}
\title{Practica 1}
\author{Por Ruben Cazorla Rodriguez}
\maketitle

\section{{\LARGE{}Ejercicio 1}}

\subsection{{\Large{}Enunciado}}

Encuentra $R^{3}$ de R = \{(1, 1),(1, 2),(2, 3),(3, 4)\}. Comprueba
tu respuesta con el script \textit{powerrelation.m }y escribe un documento
LATEX con la soluc-ion paso por paso.

\subsection{{\Large{}Resolucion}}

\ 

Para encontrar $R^{3}$ se utiliza la expresion de potencias de una
relacion

\[
R^{n}=\{(a,b):\exists x\in A,(a,x)\in R^{n-1}\wedge(x,b)\in R\}
\]

Debido a la naturaleza de la expresion necesitaremos $R^{2}$ del cual,
utilizando la expresion anterior, su conjunto quedaria asi:

\[
R^{2}=\left\{ (1,1),(1,2),(1,3),(2,4)\right\} 
\]

Y con este resultado, la solucion de $R^{3}$ seria:

\[
R^{3}=\left\{ (1,1),(1,2),(1,3),(1,4)\right\} 
\]


\subsection{{\Large{}Comprobacion}}

Utilizando el fichero \textit{powerrelation.m }en Octave, procedemos
a utilizar el comando \textit{powerrelation(l,n) }donde \textit{l
}es el conjunto de relaciones R y \textit{n }el exponente a elevar:

\ 

\textbf{powerrelation(\{{[}'1','1'{]},{[}'1','2'{]},{[}'2','3'{]},{[}'3','4'{]}\},3) }

\textbf{ans = \{ }

\textbf{{[}1,1{]} = 11, }

\textbf{{[}1,2{]} = 12, }

\textbf{{[}1,3{]} = 13, }

\textbf{{[}1,4{]} = 14 \}}

\section{{\LARGE{}Ejercicio 2}}

\subsection{{\Large{}Enunciado}}

En la carpeta ``files'', dentro del zip ``ficheros practica 1'',
encontrar un fichero de extension \textit{.tex }que contenga \textbf{\textbackslash usepackage\{asmthm,
amsmath\}}. Completa la demostracion y contesta la pregunta

\subsection{{\Large{}Busqueda}}

Utilizando el comando ``grep'' en la terminal ubicada en la carpeta
donde se encuentra ``files'' podemos encontrar el fichero que contenga
dicha cadena:
\begin{doublespace}
\noindent \begin{center}
grep \textquotedbl\textbackslash usepackage\{amsthm, amsmath\}\textquotedbl{}
files/{*} 
\par\end{center}
\end{doublespace}

\noindent Y obtenemos que el archivo \textit{mainP.tex .}

\subsection{{\Large{}Demostracion}}

Hay que demostrar que ($\alpha+\beta)\gamma=\alpha\gamma+\alpha\beta$:
\begin{center}
$L((\alpha+\beta)\gamma)=L((\alpha+\beta))L(\gamma)=(L(\alpha)\cup L(\beta))L(\gamma)=L(\alpha)L(\gamma)\cup L(\beta)L(\gamma)=L(\alpha\gamma)\cup L(\beta\gamma)=L(\alpha\gamma+\alpha\beta)$
\par\end{center}

\subsection{{\Large{}Pregunta}}

Consideramos \{$L=\{w\in\{a,b\}^{*}:w$ no termina en ab\}. Da una
expresion regular que genera L.
\begin{center}
\textbf{(a+b){*}a}
\par\end{center}
\end{document}
